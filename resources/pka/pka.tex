\documentclass[oneside]{memoir}
\usepackage{fontspec}
\usepackage[hidelinks]{hyperref}
\usepackage{microtype}
\usepackage{amsmath}
\usepackage{amssymb}
\usepackage{bm}
\usepackage{xcolor}
\usepackage{booktabs}
\usepackage{graphicx}
\usepackage[euler-digits,euler-hat-accent]{eulervm}

\setlrmarginsandblock{3.5cm}{2.5cm}{*}
\setulmarginsandblock{2.5cm}{*}{1}
\checkandfixthelayout

\definecolor{ACCgreen}{HTML}{28A745}

\newcommand\ddfrac[2]{\frac{\displaystyle #1}{\displaystyle #2}}

\setmainfont{TeX Gyre Pagella}

\usepackage{xcolor}
\hypersetup{
    colorlinks,
    linkcolor={red!50!black},
    citecolor={blue!50!black},
    urlcolor={blue!80!black}
}

\def\chapterheadstart{}
\pagestyle{empty}

\begin{document}

\begin{table}
\renewcommand\thetable{1}
\begin{tabular}{lllrr}
\toprule
\textbf{Name of the compound} & \textbf{\href{https://www.drugbank.ca/}{DrugBank ID}} & \textbf{\href{https://pubchem.ncbi.nlm.nih.gov/}{PubChem CID}} & \textbf{pKa} & \textbf{Charge on phenolic H}\\
\midrule
\href{https://en.wikipedia.org/wiki/2,4-Dinitrophenol}{2,4-dinitrophenol} &
DB04528 &
1493 &
4.09 &
0.467 \\
\href{https://en.wikipedia.org/wiki/4-Nitrophenol}{p-nitrophenol} &
DB04417 &
980 &
7.15 &
0.430 \\
\href{https://en.wikipedia.org/wiki/2-Chlorophenol}{2-chlorophenol} &
DB03110 &
7245 &
8.56 &
0.405 \\
\href{https://en.wikipedia.org/wiki/3-Chlorophenol}{3-chlorophenol} &
DB01957 &
7933 &
9.12 &
0.393 \\
\href{https://en.wikipedia.org/wiki/M-Cresol}{m-cresol} &
DB11143* &
342 &
10.10 &
0.379 \\
\href{https://en.wikipedia.org/wiki/O-Cresol}{o-cresol} &
DB11143*&
335 &
10.30 &
0.376 \\
\href{https://en.wikipedia.org/wiki/Propofol}{propofol} &
DB00818 &
4943 &
11.10 &
0.350 \\
\bottomrule
\end{tabular}

\caption{Summary information about phenolic drug compounds. *\textit{DrugBank ID links to the Cresol mixture.}}


\end{table}



\end{document}